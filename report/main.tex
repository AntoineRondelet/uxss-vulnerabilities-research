\documentclass[journal]{IEEEtran}
\usepackage{blindtext}
\usepackage{graphicx}
\usepackage[justification=centering]{caption}


\ifCLASSINFOpdf
  % \usepackage[pdftex]{graphicx}
  % declare the path(s) where your graphic files are
  % \graphicspath{{../pdf/}{../jpeg/}}
  % and their extensions so you won't have to specify these with
  % every instance of \includegraphics
  % \DeclareGraphicsExtensions{.pdf,.jpeg,.png}
\else
  % or other class option (dvipsone, dvipdf, if not using dvips). graphicx
  % will default to the driver specified in the system graphics.cfg if no
  % driver is specified.
  % \usepackage[dvips]{graphicx}
  % declare the path(s) where your graphic files are
  % \graphicspath{{../eps/}}
  % and their extensions so you won't have to specify these with
  % every instance of \includegraphics
  % \DeclareGraphicsExtensions{.eps}
\fi
% graphicx was written by David Carlisle and Sebastian Rahtz. It is
% required if you want graphics, photos, etc. graphicx.sty is already
% installed on most LaTeX systems. The latest version and documentation can
% be obtained at: 
% http://www.ctan.org/tex-archive/macros/latex/required/graphics/
% Another good source of documentation is "Using Imported Graphics in
% LaTeX2e" by Keith Reckdahl which can be found as epslatex.ps or
% epslatex.pdf at: http://www.ctan.org/tex-archive/info/
%

\begin{document}
\title{IS593 Term Project \emph{"Find Universal Cross Site Scripting Vulnerabilities in Brave Browser"}}

\author{Antoine RONDELET (20176461)}
\maketitle
\IEEEpeerreviewmaketitle

\section{Introduction}
Write the introduction at the end
Finish the introduction by a list of keywords.

It might be good to begin the paper with a bit of context about UXSS research. See whether we manage to find some papers about it -> if yes: speak about previous work, if no: speak about the fact that we couldn't find anything and that we were surprised to see that no previous work were carried out to try to study the security mechanisms of browsers and that current UXSS investigations are carried out by either companies themselves or by isolated persons reporting bugs to companies. This paper is an attempt to study browser's vulnerability, especially Universal XSS vulnerabilities.

Maybe do a small introduction about us (the team) and say that we are students in the computer science and information security departments at KAIST. Passionate by security and cryptography we decided to study UXSS. We have a very small background in Web programming and didn't know that much about the way browsers worked before this study, however, the potential theat behind UXSS attacks attracted our attention, and further to some research, it turned out that UXSS investigations were carried out by either companies themselves or by isolated persons reporting bugs to companies. Thus, our paper is an first attempt to study browser's vulnerability, especially Universal XSS vulnerabilities, and gather what we found. 

\section{Web Browsers}

\subsection{Role}
Before presenting our work, we want to define what exactly a browser is and what is its intrinsic role. \\
According to Wikipedia, a web browser (commonly referred to as a browser) is a software application for retrieving, presenting and traversing information resources on the World Wide Web.\footnote{https://en.wikipedia.org/wiki/Web\_browser} 
While the primary role of these pieces of software is to provide an easy way for users to access web resources, most of today's web browsers implement a myriad of features such as the support for secure communications, the possibility to execute Javascript code or even the ability to be extended with plugins, to only name a few.

\begin{figure}[h]
\includegraphics[width=0.4\textwidth]{images/WebBrowserMarketShare.png}
\caption{Web Browser Market Share in October 2017 \\ (from www.w3counter.com)}
\label{fig:marketShare}
\end{figure}

Around two decades after Netscape's Navigator and Window's Internet Explorer \emph{Browser wars}, the competition is still prevailing on the web browser market. The democratization of the use of the Web and, more recently, the introduction of HTML5, CSS3, and extensive client-side scripting to the World Wide Web, as well as more widespread use of smartphones and other mobile devices for browsing the web, lead to the apparition of dozens of browsers. From Mozilla's Firefox\footnote{https://www.mozilla.org/en-US/firefox/}, Google's Chrome\footnote{https://www.google.com/chrome/index.html}, Apple's Safari\footnote{https://www.apple.com/safari/} to Opera\footnote{http://www.opera.com/fr} for the most famous, the number of new browsers never ceases to increase. Today, as we are moving to the 4th industrial revolution - where data is moving astoundingly fast, and collected massively to be fed into \emph{Intelligent Agents} - some newcomers such as Brave\footnote{https://brave.com} or even Whale\footnote{http://whale.naver.com/en/} are trying to enter this disputed market and get their slice of the cake.
Although, figures \ref{fig:marketShare} and \ref{fig:monthlyUsage} agree on the fact that Google Chrome is the most widespread browser, other companies are working hard and hiring massively to try to dethrone Google's super star.

\begin{figure}[h]
\centering
\includegraphics[width=0.4\textwidth]{images/BrowserFamilyMonthlyUsageShare.png}
\caption{Web Browser Monthly Usage Share \\ (from www.w3counter.com)}
\label{fig:monthlyUsage}
\end{figure}


As opposed to Netscape's Navigator, today, most web browsers are free to use. Developing web browsers is not about making money anymore, but more about sculpting people's computing habits in a way that would benefit the companies behind them.
As the market is quite competitive, developing a web browser can be seen as taking part into a race to developing the most features. This willingness to be different and attract people's attention often lead web browsers developers to implement features in a hurry. Such frequent releases sometimes unveil exploitable vulnerabilities and make browser's architecture more and more complex.

\medskip

A taste of a web browser's design is provided in the next part of this paper.

\subsection{Architecture}

Despite the large variety of browsers available to surf the web, most of them tend to follow a general structure. This architecture, as shown in \cite{architectureWebBrowsers} \cite{howBrowsersWork} can be decomposed into eight components, each of which has its very own functionality.

\subsubsection{User Interface}
The component in charge of the UI contains every piece of the browser that is not actually part of the webpage itself. Such piece can either be toolbars or Back and forward buttons for instance.

\subsubsection{Browser Engine}
The browser engine forms the interface that allows the User Interface to interact and manipulate elements of the rendering engine. It basically marshals actions between the UI and the rendering engine.

\subsubsection{Rendering Engine}
The rendering engine is responsible for producing a graphical representation of the document fetched from a specific URL. It parses and render documents written in HTML and applies styles specified by CSS.

\subsubsection{Data Persistence}
The data persistence subsystem is responsible for insuring the persistence of data when the browser needs it. For instance, a browser needs to save data locally to keep track of the user's bookmarks and cookies.

\subsubsection{Networking}
Since a browser fetches document from the Web, a browser needs to be able to have an access to the network. This access is provided by the networking module. It implements file transfer protocols such as HTTP and FTP. Moreover, this subsystem translates between different character sets, and resolves MIME media types for files\footnote{https://en.wikipedia.org/wiki/Media\_type}. In some cases, the networking module might also implement a cache of recently retrieved resources to ensure good performances when a user asks multiple times the same resource.

\subsubsection{Javascript Interpreter}
This module parses and executes Javascript\footnote{Programming language developed by Netscape in 1995 to make webpages interactive. More details can be found at https://en.wikipedia.org/wiki/JavaScript} code.

\subsubsection{XML parser}
The XML parser module is in charge of parsing XML documents into a Document Object Model (DOM) tree\footnote{https://en.wikipedia.org/wiki/Document\_Object\_Model}

\subsubsection{Display Backend}
The display backend subsystem provides primitives for drawing and windows. It also interfaces with the Operating System of the host machine.

\begin{figure}[h]
\centering
\includegraphics[width=0.4\textwidth]{images/BrowserStructure.png}
\caption{General Web Browser Structure}
\label{fig:browserStructure}
\end{figure}

While figure \ref{fig:browserStructure} shows the general web browser structure explained above, it is without saying that each browser has its particularities and its very own version of the aforementioned architecture.

\medskip

At this point, the reader should be able to have a taste of the complexity of web browsers. Such a compound of subsystems might be likely to expose several vulnerabilities either in the subsystems' implementation or in their interactions.

\subsection{Security mechanisms}
While, the Internet today is a widespread information infrastructure, the Web is not only used to access and share resources, but is also hosting commercial activities. According to http://www.internetworldstats.com/stats.htm, 51.7\% of the world population\footnote{This represent 3,885,567,619 people} is now using the Internet on a day to day basis. Albeit, a vast majority of Internet users are undoubtedly honest users, a few of them are malicious and carry out attacks on the network or publish malicious web pages on the web waiting for users to fetch them. From this point, a malicious page displayed in victims' browser could run malicious Javascript code trying to retrieve user's sensitive data. \\
In fact, in general, many pages tend to be displayed in a user's browser at the same time (into tabs or iframes\footnote{https://developer.mozilla.org/en-US/docs/Web/HTML/Element/iframe}). Thus, they all live in the browser in the same time. So how about using the browser as a bridge for a malicious page to access sensitive data stored in other pages ? \\

Since browsers are the door that enable us to access web pages from a lot of sources every day, such a vulnerability would be a major threat for Internet users. Being able to keep browsers safe from such menaces is paramount to keep the web safe. In order to do so, web browsers developers developed very sophisticated security mechanisms to isolate web pages from one another.

Moreover, modern web browsers need to support the latest and constantly evolving web standards capable of loading resources, while making sure that their level of security remains optimal. As a matter of fact, as explained in \cite{browserSecurity}, web browsers are committed to implement most of the security specifications defined by the World Wide Web Consortium (W3C), such as: (see \cite{SOP} \cite{CORS} \cite{CSP} \cite{SRI} \cite{MixedContent})

\medskip

\subsubsection{Same Origin Policy (SOP)}

The foundation of browsers security mechanisms is the notion of Same Origin Policy (referred to as SOP). Before looking at this important concept, we should first define what an origin is.

\medskip

In order to access a specific resource, a user has to specify its Uniform Resource Locator (URL)\footnote{Reference to a web resource that specifies its location on a computer network and a mechanism for retrieving it.}. This web address is a compound of different elements, as shown on figure \ref{fig:URL}. The protocol indicates which protocol the browser must use to retrieve the resource. The domain name is an alias for the IP address pointing to the Web server that should be requested. The port is a logical construct that identifies a specific process or a type of network service, on the web server, that is responsible for serving the web resource. Moreover, the path, parameters and fragment are the path to the resource on the Web server, extra parameters provided to the Web server and an anchor to a specific part of the resource itself, respectively.

\begin{figure}[h]
\centering
\includegraphics[width=0.4\textwidth]{images/URL.png}
\caption{URL anatomy}
\label{fig:URL}
\end{figure}

\medskip

With this in mind, defining what an origin is becomes trivial. Looking at the official definition of the W3C\footnote{https://www.w3.org/Security/wiki/Same\_Origin\_Policy}, an origin is defined by the scheme, host, and port of a URL.

\medskip

Know that we know what an origin is, the Same Origin Policy is a policy that ensures that documents retrieved from distinct origins are isolated from each other. Thus, even if users happened to visit malicious web sites, these fraudulent sites would not be able to interfere with the user's session with honest web sites.

\begin{figure}[h]
\centering
\includegraphics[width=0.4\textwidth]{images/SOPTabs.png}
\caption{Same Origin Policy preventing a malicious "tabs" to access user's sensitive data}
\label{fig:SOPTabs}
\end{figure}

\subsubsection{Cross-Origin Resource Sharing (CORS)}

In order to relax the SOP, the W3C specified the Cross-Origin Resource Sharing (CORS) which consists in specifying a white-list of trusted domains allowed to request restricted resources, by extending HTTP with a new origin request header. The CORS mechanism supports secure cross-domain requests and data transfers between browsers and web servers. \\

Even though, the policy is defined by web server administrators, this security mechanism is used in modern browsers especially on APIs such as \emph{XMLHttpRequest} or \emph{Fetch} to help mitigate the risks of cross-origin HTTP requests and prevent a malicious user to access a resource he should not be able to access.

\subsubsection{Content Security Policy (CSP)}

The Content Security Policy allows web site administrators to define a white-list in a HTTP header to specify trusted sources for delivering content. That way web authors can control resources the user agent is allowed to load for a given page.

\subsubsection{Mixed Content Blocking}

The mechanism of Mixed Content Blocking blocks insecure content on web pages that are considered secure. A "mixed" content can happen in the case where a user fetches a secure page over HTTPS, containing a HTTP content. Since data transiting over HTTP is not secure and can be eavesdropped, the Mixed Content Blocking mechanism will block this part of the web page. That way the page displayed to the user would be "completely safe from risks"\footnote{This section is placed into quotes because in web security, very few things (not to say nothing) are said to be unconditionally safe.}.

\subsubsection{Subresource Integrity (SRI)}

Subresource Integrity defines a mechanism by which user agents may verify that a fetched resource has been delivered without being altered. This process allow a User Agent to be sure that the requested resource is the one it received. SRI works by using hash functions. Website developers can associated a hash value to a specific resource. That way web browsers can compute the hash value of the received resource and compare them with the initial value provided by the website. The resource is then displayed only if the hash values are equal, i.e: the resource is valid. 

\medskip
\medskip

Despite the implementation and support of the security specifications of the W3C, some lower level attacks could be carried out to escape the security mechanisms of today's web browsers. By compromising a browser process and execute arbitrary code, and attacker could either be able to access the host system or bypass the SOP. The next subsections present the concepts of Sandboxes and Process and Origin Isolation, that are currently ensuring that low level attacks would not have dramatic impacts for end-users, and that SOP would not be "bypassable" at lower levels.

\subsection{Sandboxes}

Sandboxes is a mechanism which consists in limiting vulnerabilities by isolating components of an application from each other and from the rest of the system. In a sandbox, components run with the minimum access privileges to system resources they need to perform their functions.
By isolating complex components into sandboxes, web browser developers can diminish the impact vulnerabilities can have for the end-user. By limiting the damages an attacker can cause. \\

Some additional pieces of information about Google Chrome's sandboxes are provided in \cite{browserSecurityWhitePaper}, and an extract is illustrated in figure \ref{fig:ChromeSandboxes}.

\begin{figure}[h]
\centering
\includegraphics[width=0.4\textwidth]{images/SandboxesChrome.png}
\caption{Left: Google Chrome Main Process Sandbox, Right: Google Chrome Render Sandbox}
\label{fig:ChromeSandboxes}
\end{figure}

\subsection{Process and Origin Isolation}

We studied the mechanism of Same Origin Policy earlier in this paper. This concept restricts access to data across webpages if they have different origins. While the SOP is generally implemented at a high level, an attacker able to achieve process-level attacks might be able to bypass the Same Origin Policy, and thus accessing resources of a different origin hosted in the same process. To void such scenarios, decided to host each origin in a different process. That way by preventing direct access between these processes, and applying the same-origin policy in the APIs used by these processes for inter-origin communications, the same-origin policy can be enforced at all levels in browsers. The idea here is to ensure a strong site isolation and establish strong security boundaries between web sites \cite{isolationChrome}.

\begin{figure}[h]
\centering
\includegraphics[width=0.4\textwidth]{images/IsolationChrome.png}
\caption{Site isolation in Chrome}
\label{fig:ChromeIsolation}
\end{figure}


Bypassing a browsers same-origin policy generically is called Universal Cross-site Scripting (UXSS).

\section{Universal XSS}

\subsection{\emph{Classic} XSS attack}
Explain what XSS is. They target WEB APPS

\subsection{UXSS a major threat}

Universal XSS (UXSS) is a particular type of Cross-Site Scripting that has the ability to be triggered by exploiting flaws inside browsers, instead of leveraging the vulnerabilities against insecure web sites.
In UXSS attacks, client-side vulnerabilities are exploited in a web browser or browser extensions to generate an XSS condition, which allows the malicious code to be executed, bypassing or disabling the security protection mechanisms in the web browser.

Describe what Universal XSS is, and explain the consequences it has. It targets the BROWSER and could defeat all protection mechanisms of web apps.

\section{Brave}
Speak about the new features brought by Brave (speak about the company's policy to ensure privacy on the web and so on...) + the BAT principle and the new things that are brought by Brave and why we decided to study this web browser in particular
+ Speak about other new browsers such as Whale and so on just to show that many actors are actually trying to take Google Chrome's market share.

Say that we first focused on studying Brave because of X reasons, and that we broaden our research to other browsers.

\section{Previous attacks and patterns used to defeat the security mechanisms of web browsers}
Today web browsers are extremely complicated pieces of software (speak about complex architecture described in the section above). So there are potential vulnerabilities that escaped to the development teams and can be used by malicious users.

- Say that we observed many kind of attacks. A UXSS vulnerability can be found in different subsystems of the browser or in plugins/extensions (example of the Adobe Reader plugin \cite{uxssPDF} and Keyabse \cite{uxssKeybase})

\medskip

Speak about the patterns we found in the different attack report we studied (CVE and Broken Browser and so on)

\section{Our attempts}
Say that none of our attempts were successful but that we provide them as a record of our work. They might also be extended/adapted to carry out successful attack.

\medskip

Speak about what we tried, and why we tried it, and explain what we had in mind when we tried something (so that the reader can understand the attack more easily)

\medskip

Conclude this section by saying that today's browsers are not trivially vulnerable (fortunately), however it is not because we didn't manage to find a vulnerability that no one is going to find one.

\section{Conclusion}
Despite the fact that we didn't manage to find any vulnerability on the web browsers, some persons might be able to find some. It is important to remember that browsers act as a last rampart against malicious users/attackers. As many browsers are on the market today, the teams developing them are in a perpetual race against each other. The best browser is the most secure one with the most features to provide their users. However, as features are added to a browser, it is paramount to keep in mind the aspect of security.

\section{IDEAS for the paper}
Note, while we were writing this paper, Mozilla released a brand new version of Firefox (Firefow Quantum). We didn't have time to test malicous payloads against this browser.



% if have a single appendix:
%\appendix[Proof of the Zonklar Equations]
% or
%\appendix  % for no appendix heading
% do not use \section anymore after \appendix, only \section*
% is possibly needed

% use appendices with more than one appendix
% then use \section to start each appendix
% you must declare a \section before using any
% \subsection or using \label (\appendices by itself
% starts a section numbered zero.)
%

% Can use something like this to put references on a page
% by themselves when using endfloat and the captionsoff option.

\begin{thebibliography}{1}

\bibitem{website}
https://www.brokenbrowser.com/uxss-edge-domainless-world/

\bibitem{paper1}
MJ Rees,
\textit{Evolving the Browser Towards a Standard User Interface Architecture}, 2001

\bibitem{DOM}
W3C,
\textit{Document Object Model (DOM)},
https://www.w3.org/DOM/

\bibitem{architectureWebBrowsers}
Alan Grosskurth, and Michael W. Godfrey,
\textit{A reference architecture for web browsers}, 2006

\bibitem{howBrowsersWork}
Tali Garsiel and Paul Irish,
\textit{How Browsers Work: Behind the scenes of modern web browsers}, 2011

\bibitem{browserSecurity}
Christoph Kerschbaumer,
\textit{Enforcing Content Security by Default within Web Browsers}, IEEE, 2016

\bibitem{SOP}
W3C,
\textit{Same Origin Policy (SOP)},
https://www.w3.org/Security/wiki/Same\_Origin\_Policy

\bibitem{CORS}
W3C,
\textit{Cross-Origin Resource Sharing (CORS)},
https://www.w3.org/Security/wiki/CORS

\bibitem{CSP}
W3C,
\textit{Content Security Policy (CSP)},
https://www.w3.org/TR/CSP/

\bibitem{SRI}
W3C,
\textit{Subresource Integrity},
https://www.w3.org/TR/SRI/

\bibitem{MixedContent}
W3C,
\textit{Mixed Content},
https://www.w3.org/TR/mixed-content/

\bibitem{browserSecurityWhitePaper}
Markus Vervier, Michele Orrù, Berend-Jan Wever, Eric Sesterhenn,
\textit{Browser Security White Paper},
X41 D-SEC GmbH, 2017
 
 
\bibitem{browserSecurityWhitePaper2}
Mario Heiderich,
\textit{Browser Security White Paper},
Cure53, 2017

\bibitem{isolationChrome}
The Chromium Projects,
\textit{Site Isolation},
https://www.chromium.org/developers/design-documents/site-isolation

\bibitem{uxssPDF}
Ofer Shezaf,
\textit{The Universal XSS PDF Vulnerability},
OWASP IL Chapter leader

\bibitem{uxssKeybase}
\textit{Universal Cross-Site Scripting in Keybase Chrome extension},
https://hackerone.com/reports/232432
 
\bibitem{webSecurityFoundation}
Devdatta Akhawe, Adam Barth, Peifung E. Lam, John Mitchell and Dawn Song,
\textit{Towards a Formal Foundation of Web Security}
 
\bibitem{javascriptSecurity}
Nataliia Bielova,
\textit{Survey on JavaScript security policies and their enforcement mechanisms in a web browser}
The Journal of Logic and Algebraic Programming, 2013

\end{thebibliography}

\ifCLASSOPTIONcaptionsoff
  \newpage
\fi



% trigger a \newpage just before the given reference
% number - used to balance the columns on the last page
% adjust value as needed - may need to be readjusted if
% the document is modified later
%\IEEEtriggeratref{8}
% The "triggered" command can be changed if desired:
%\IEEEtriggercmd{\enlargethispage{-5in}}

% references section

% can use a bibliography generated by BibTeX as a .bbl file
% BibTeX documentation can be easily obtained at:
% http://www.ctan.org/tex-archive/biblio/bibtex/contrib/doc/
% The IEEEtran BibTeX style support page is at:
% http://www.michaelshell.org/tex/ieeetran/bibtex/
%\bibliographystyle{IEEEtran}
% argument is your BibTeX string definitions and bibliography database(s)
%\bibliography{IEEEabrv,../bib/paper}
%
% <OR> manually copy in the resultant .bbl file
% set second argument of \begin to the number of references
% (used to reserve space for the reference number labels box)

% biography section
% 
% If you have an EPS/PDF photo (graphicx package needed) extra braces are
% needed around the contents of the optional argument to biography to prevent
% the LaTeX parser from getting confused when it sees the complicated
% \includegraphics command within an optional argument. (You could create
% your own custom macro containing the \includegraphics command to make things
% simpler here.)
%\begin{biography}[{\includegraphics[width=1in,height=1.25in,clip,keepaspectratio]{mshell}}]{Michael Shell}
% or if you just want to reserve a space for a photo:

% You can push biographies down or up by placing
% a \vfill before or after them. The appropriate
% use of \vfill depends on what kind of text is
% on the last page and whether or not the columns
% are being equalized.

%\vfill

% Can be used to pull up biographies so that the bottom of the last one
% is flush with the other column.
%\enlargethispage{-5in}



% that's all folks
\end{document}


